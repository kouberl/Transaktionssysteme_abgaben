\documentclass{scrartcl}
\usepackage[utf8]{inputenc}
\usepackage{enumerate,amsmath}

\title{Transaktionssysteme -- Übungsblatt 1}
\subtitle{Gruppe 1, Team Zoidberg}
\author{Koppera Thomas \\ 51366
 \and Kinseher Josef \\ 
 \and Treuberg Matthias \\ 51164}
\date{ }

\begin{document}
  \maketitle
  
\section*{Aufgabe 1}
\begin{enumerate}[a)]
\item
\begin{itemize}
\item
Atomizität (Atomicity)\\
Transaktionen sind atomar, d.h. Transaktionen werden entweder ganz
oder garnicht ausgeführt.
\item
Consistenz (Consistency)\\
Transaktionen überführen einen korrekten Systemzustand wieder in
einen korrekten Systemzustand.
\item
Isolation\\
Transaktionen werden durch die Existenz anderer (parallel laufender)
Transaktionen nicht beeinflusst und beeinflussen diese nicht.
Insbesondere "'sehen"' Transaktionen nur konsistente Daten.
\item
Dauerhaftigkeit (Durability)\\
Wirksam gewordene Änderungen, die von einer Transaktion veranlasst
worden sind, sind dauerhaft, d.h. sie überdauern auch Fehler beim
Datenserver oder im Kommunikationsnetz.
\end{itemize}

\item
\begin{itemize}
\item
Beispiel Banktransaktion: Einem wird Geld abgezogen, der andere bekommt aber wegen Systemcrash das Geld nicht.
\item
Beispiel Bank: Kontostand besteht nach Transaktion aus einem "'Buchstabenwert"'.
\item
Beispiel Bank: Eine Transaktion wird nicht ausgeführt, weil bereits eine andere Banktransaktion ausgeführt wird.
\item
Beispiel Bank: Bankkonto leer nach Crash.
\end{itemize}


\end{enumerate}

\section*{Aufgabe 2}
\begin{enumerate}[a)]
\item MyISAM kann keine der vier ACID-Eigenschaften garantieren.
\item Isolation
\item Wenn ich mich nur auf Lesezugriffe beschränke.
\item Bei kleinen Datenbanken oder Anwendungen wo viel und oft Daten geändert werden: Wordpress, Joomla
\item Bei größeren Projekten bei denen Sicherheit vor Performance steht.

\end{enumerate}

\section*{Aufgabe 3}

\begin{enumerate}[a)]
\item
\item
\end{enumerate}

\end{document}
